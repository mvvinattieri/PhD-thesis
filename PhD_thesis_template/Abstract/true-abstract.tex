\chapter*{Abstract}
We are interested in a specific aspect related to modeling disease development, i.e. the study of family-specific risk. Generally, quantifying the family-specific risk of developing a disease is crucial to improve the patient’s survival, and healthcare data provide an essential tool to estimate such quantity. Specifically for breast cancer, early detection is particularly important to increase the chances of a successful treatment. In this setting, risk estimation allows for the identification of subjects at higher risk of developing breast cancer, so that they can be the target of tailored
and more intensive screening and prevention strategies. We take families as our statistical units of interest, and we assume that these units belong to different risk groups of developing the disease. We investigate models on the age at breast cancer onset, with a structure of familiar dependence among survival times through the risk.
In Chapter 1, we compare cure rate models with a continuous Gamma frailty to the conventional Cox model in terms of risk prediction accuracy. In Chapter 2, however, we focus exclusively on cure rate models using a binary frailty to model the age at onset. In Chapter 3 we move to the study of the heritability of longevity. Similarly to the previous chapters, we assume that families have different “risk” in terms of life expectancy. We carry out our analysis through simulation studies and apply the continuous frailty model from Chapter 1 to the available data sourced from the Multi-Generational Breast Cancer Swedish registry. We conclude with the exploration of most powerful tests on survival data with right censoring in Chapter 4.