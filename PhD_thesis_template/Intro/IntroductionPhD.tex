\chapter*{Introduction}
We focus on a specific aspect of modeling the disease development, namely the investigation of family-specific risk. Generally, quantifying the family-specific risk of developing a disease is crucial to improve early detection of the disease and consequently the patient’s survival. Healthcare data provide an essential tool to estimate the family-specific risk, and thus we develop and employ specific methods to estimate this risk from the available data. We focus on breast cancer onset,  although the problem can be generalized to a wide range of diseases. In the context of breast cancer development, risk estimation plays a crucial role in identifying families, and thus individuals, who are at the highest risk of developing breast cancer. We assume that individuals who belong to the same family share a common risk of being diagnosed with breast cancer from birth. Thus, this family-specific risk is unchanged from birth and never observable. Because of the latent nature of the risk we call it the frailty risk. Once we identify the highest-risk families, that are those with highest values of the frailty risk, we can target family´s female members for tailored and more intensive screening and prevention strategies, which can improve their chances of a successful treatment and prognosis.

One common approach to split families into different risk groups of developing breast cancer is considering the strongest risk factors associated to breast cancer development. In our motivating data, one among the strongest risk factors is available: the breast cancer family history.  The breast cancer family history is defined as the indicator of observing at least one family member who has already experienced breast cancer diagnosis at a fixed time. Specifically, the family history indicator $FH(t)$ takes value one when, at time $t$, at least one family member has been diagnosed with invasive breast cancer, otherwise it takes value $0$. This indicator is based on the notion that individuals with a positive (i.e. $FH(t) = 1$) family history are at a higher risk of developing breast cancer itself. From one side it is convenient to involve this easy indicator into a model with the aim of splitting families into groups, that are in total two: the low-risk group and the high-risk group of developing breast cancer. Also, this approach motivates the categorization of families into two frailty risk groups, allowing the frailty risk to be binary with two levels $0/1$, for low and high-risk groups respectively, as we illustrate in Chapter \ref{chapter:2}. On the other side the family history, and consequently the binary frailty risk, cannot capture the complex nature of such a phenomenon as breast cancer development. This motivates our primary novel contribution of this thesis that is the extension of the frailty risk to be continuous with infinitely many values, as we illustrate in Chapter \ref{chapter:1}.  Specifically, the continuous frailty risk is assumed to be distributed according to a Gamma distribution with parameters $(shape=\theta, \ rate=\theta)$, where we call $\theta$ the frailty parameter. Once we assume a continuous frailty risk we carry out a comparative analysis to assess that our proposed model, namely the Multivariate frailty Cure-Rate model, outperforms among the other models under analysis in terms of inference precision, prediction accuracy, and explanability of the phenomenon. The multivariate part of our proposed model refers to jointly model the family female members time-to-events (where the event of interest is breast cancer onset in our case) in contrast to modelling only one subject per family, as the Univariate frailty Cure-Rate model or the Univariate FH Cure-Rate model do, as we develop and see in Chapter \ref{chapter:1}.  While, the Cure-Rate component refers to the peculiar structure that the survival function takes based on the assumption that not all the women will experience breast cancer onset eventually in their lifetime. The proportion of women that won´t experience the breast cancer onset is called the cured fraction. The cured fraction enters then the Cure-Rate survival function formula, making the survival function not proper anymore as we will see deeper into Chapter \ref{chapter:1}. Thus, through the Multivariate frailty Cure-Rate model our novel contribution consists of involving the Cure-Rate baseline survival function in a Lehmann structure that unites the different family-specific frailty risk survival functions. In other words, each family has a specific survival function that depends on the Cure-Rate baseline survival function and on its frailty risk values through the Lehmann structure. Moreover, we prove that also the family-specific frailty survival function follows a Cure-Rate structure itself.  

Hence, the Multivariate frailty Cure-Rate model allows us to estimate the survival distribution of  ``cases'', that are defined as those women that will eventually experience the breast cancer onset in their lifetime, and to capture a peculiar tail behaviour by estimating the cured fraction. These measures accurately contribute to explain the magnitude of breast cancer development into the population of interest. In contrast, for example, a more flexible model as the already known and developed Multivariate frailty Cox model would not allow us to estimate such quantities. Moreover, we show that Multivariate frailty Cure-Rate model achieves a level of prediction accuracy on par to the Multivariate frailty Cox model. This highlights the superior potential of our proposed model over the traditional Cox model, leaving no room for uncertainty that the Multivariate frailty Cure-Rate model strikes a balance among precision in inference and explainability of the phenomenon without losing points about prediction accuracy. 

% Our main assumption is that , and we must employ appropriate methods to deal with the problem of missing information due its latency. 

% To achieve this goal, we take families as our units of interest and assume that they have a value of the frailty risk among the infinitely many in Chapter \ref{chapter:1}. While in Chapter \ref{chapter:2} we assume that families can be categorized into two latent risk groups based on their risk of developing the disease, thus the risk $R$ is allowed to be binary with two levels $0/1$, for low and high-risk groups respectively.

% The main raw purpose here is to estimate the frailty parameter, that is necessary for computing the posterior family-specific risk of breast cancer occurrence. The estimation of the frailty parameter can occur through several implemented models. We explore a total of six models, with three of them that are semiparametric, and three of them that are parametric. Specifically, we implement a model that we call ``multivariate'', meaning that the complete family information from all the family members on breast cancer history is jointly model to accurately estimate the frailty. Two multivariate model are implemented: our frailty parametric model and the semiparametric Cox model. The other models are called ``univariate'' to indicate that only one subject per family is involved. One of the them is a univariate frailty model, while the other tree replace the unknown frailty quantity with a summary of the family history (FH) of breast cancer as a covariate.

% With the estimate of the frailty parameter we can estimate the poster family-specific risk distribution, and thus the risk can be estimated as the posterior mean, median, or mode. For ease of interpretation and better performances in prediction accuracy, we select the mean as frailty estimate. 

% From the analysis, we expect that the multivariate models perform better in terms of risk prediction accuracy compared to the other models, since they are jointly modelling all the familiar information. Moreover, we expect that the parametric multivariate model performs better than its semiparametric counterpart in terms of precision in inference, prediction accuracy, and interpretability of the phenomenon.

% Notably, let us say that all the parametric models that we implement follow a cure rate survival structure which assume that not all women will develop breast cancer in their lifetime, in contrast to the traditional survival model which assumes that all individuals will experience the event eventually. We believe that the cure rate structure is more realistic as a representation of breast cancer development, as it allows for the possibility of some individuals not to develop the disease. This is an additional reason why we believe the parametric multivariate model to be better then its semiparametric counterpart because it can estimate the cured fraction that accurately contributes to explain the magnitude of the phenomenon of worldwide interest that is the breast cancer development.

% After preliminary simulation studies, a real analysis has been carried out on an outstanding dataset, based on the Multi-generational Swedish Breast Cancer registry, that Kamila Czene from the Medical Epidemiology and Biostatistics (MEB) department of Karolinska Institutet in Stockholm, Sweden provided to us. 

Similarly we obtain in Chapter \ref{chapter:2} where, as already mentioned above, the family history approach motivates the use of a binary frailty risk to split families into two risk groups. Indeed here, the comparison between the Multivariate frailty Cure-Rate model and the Univariate frailty Cure-Rate to the Univariate FH Cure-Rate model is more meaningful than in the previous chapter thanks to the same binary nature of the frailty risk and the risk factor used as its replacement. Nevertheless, the Univariate FH Cure-Rate model may be not accurate in explaining the breast cancer development. And so may be the Univariate frailty Cure-Rate model. From results, the Multivariate frailty Cure-Rate model outperforms the other two models in terms of accuracy in risk prediction. 

Modelling a continuous frailty risk allows us to capture the complexity of the phenomenon of the breast cancer development and deal with this problem in a real dataset, beyond simulation studies. On the other side, modelling a binary frailty risk might seem simpler and more direct, but with the drawback of not capturing the real differences in risk among families, moreover in a large dataset as we have. Let us say that splitting families into two categories and identifying the higher-risk families remains clinically convenient. That is why we have included a section in Chapter  \ref{chapter:1} discussing the transition from infinitely many values of the risk among families to only two values of the risk (low and high) by splitting the families according to a fixed threshold. 

One strength of this thesis is the use of the Cure-Rate survival function in contrast to the traditional survival function that tends to 0 when the time-to-event tends to $+\infty$, meaning that all individuals will experience the event of interest eventually. For sake of completeness, we want to study also the use of a traditional survival function on the event death, as we firmly believe that with another event such as disease onset we should use a Cure-Rate survival function. On the other side, not to be too far from the study of the family-specific frailty risk of developing breast cancer, we move to the study of family-specific frailty risk of mortality in Chapter \ref{chapter:3} or, in more optimistic words, the heritability of longevity within families. Indeed, we believe that families belong to different groups of life expectancy according to their common genetics (internal factors) and life-habits (external factors). Similarly to the other two chapters, the focus in on posterior risk prediction through summarizing the posterior risk distribution. The novel contribution here is the development of a classification algorithm that operates to compute the posterior distribution of the risk of mortality. The algorithm implemented is applied to scenarios involving both discrete k-level risk or continuous risk. Additionally, we explore the binary splitting as an alternative approach to identify families into two risk groups, where one has higher life expectancy than the other. This also resembles the framework of Chapter \ref{chapter:1} and Chapter \ref{chapter:2}. Hence, it is of interest how, beyond contributing to the explanation of heritability of longevity, there is the chance to link mortality to disease onset by addressing families with the lowest life expectancy to strategies of clinical prevention of diseases and also improvement of their life habit routine.  

One notable aspect of this thesis lies in its expansion of the Lehmann family to the Cure-Rate models. While our primary focus has been on Cure-Rate models thus far, the concluding and most theoretically rich chapter in this thesis delves into the application of the Lehmann structure to proportional hazards models. Here, we delve into the traditional survival functions, and the prospect of extending our work to include Cure-Rate models emerges as a direct and intriguing extension. Thus, the exploration of most powerful tests for survival data with and without right censoring has been carried out in Chapter \ref{chapter:4}. Under the proportional hazards assumption we aim to evaluate whether an independently and identically distributed (i.i.d.) sample from a population exhibits survival times that are governed by a known survival function denoted as $S_0(t)$, rather than being governed by an unknown survival function denoted by $S_1(t)=[S_0(t)]^{\beta}$, with $\beta$ not equal to one. 

To better understand aims, findings and connections between the chapters of the thesis I proposed two tables (Tables \ref{table:1}, and \ref{table:2}) which briefly summarize all the work done.

\textcolor{blue}{\begin{center}
\begin{table*}[!h]%
\caption{Connection among the chapters.}
\begin{tabular*}{\textwidth}{@{\extracolsep\fill}lllll@{}}
    \toprule
    &Lehmann &Survival function &Event of interest &Frailty risk \\ \midrule 
    Chapter 1 &\cmark &Cure-Rate &Breast Cancer onset &Continuous \\
    Chapter 2 &\cmark &Cure-Rate &Breast Cancer onset &Binary \\
    Chapter 3 &\cmark &Traditional &Death &Discrete/Continuous \\
    Chapter 4 &\cmark &Traditional &Disease onset &\xmark \\ \bottomrule
    \end{tabular*}
\label{table:1}
\end{table*}
\end{center} }

\begin{center}
\begin{table*}[!h]%
\caption{Aims and findings.}
\begin{tabular*}{\textwidth}{@{\extracolsep\fill}ll@{}}
    \toprule
    &Aims and findings \\ \midrule 
    Chapter 1 &The Multivariate frailty Cure-Rate model outperforms the others. \\
    Chapter 2 &The Multivariate frailty Cure-Rate model outperforms the others. \\
    Chapter 3 &An algorithm for risk prediction has been developed. \\
    Chapter 4 &An application of the Lehmann family to PH models has been developed. \\ \bottomrule
    \end{tabular*}
\label{table:2}
\end{table*}
\end{center} 


